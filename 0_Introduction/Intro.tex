\section*{Introduction}

Cryptocurrencies, served as a new digital asset, have attracted much attention from both investors and academics. Along with this growing popularity, the market capitalization of cryptocurrencies is increasing substantially. Thus, according to a recent report \citep{TransparencyMarketResearch.2018}, the total capitalization for cryptocurrencies market was around US\$574.3 mn in the year 2017 and is expected to reach US\$6702.1 mn by the end of 2025. Most articles focus on Bitcoin as it is considered the first decentralized cryptocurrency, which has the largest capitalization from its beginning till now. An extensive review of the literature regarding the Bitcoin (BTC) can be found in \cite{Corbet.2018}. \hyperref[table:intro_sources]{Table 1} lists the a synthesis of the empirical findings regarding the statistical properties of the cryptocurrencies, compared to classical assets.
 
Given preceding results from the literature, our contribution to the studies dealing with the cryptocurrencies market is mostly empirical, proving the validity of phenotypic convergence in case of cryptocurrencies. In biology, the phenotype of an organism \citep{Mahner.1997} is the set of the organism's observable characteristics (i.e. morphology, developmental processes, biochemical and physiological properties). Phenotypic convergence or convergent evolution can be defined as "the appearance of similar phenotypes in distinct evolutionary lineages" \citep{Washburn.2016}. If the assets universe is regarded as an ecosystem, then we can construct an analogy with the biological concepts. Thus, the phenotype of an asset can be determined by some statistical features of the time series of price or returns. In order to derive the assets phenotype, we are using a genus-differentia approach, allowing to separate the cryptocurrencies universe from the classical assets.

Our approach is quite different from the existing literature, as most of the reviewed paper are using a low-dimensional approach when trying to differentiate cryptocurrencies from classical assets, by using either market risk indicators or long memory indicators. By using a multidimensional approach and taking into account various statistics describing the tail, moment and memory behaviour of the time series of daily log-returns, we find the proximal genus and the specific difference (\textit{genus proximum et differentia specifica}) of the daily time series of cryptocurrencies returns.
\begin{table}[H]
	\tiny{
\caption{: Empirical findings on the cryptocurrencies market}
\label{table:intro_sources}}

\tiny{
\begin{tabularx}{\textwidth}{p{2cm}|p{4cm}|p{1.5cm}|p{6.3cm}|}
\hline \hline
Authors & Assets & Sample & Findings \\
\hline
\cite{AnneHauboDyhrberg.2016} & BTC, USD/EUR, USD/GBP, FTSE index. & 2010-2015, daily data. & BTC can act as a hedge between UK equities and USD. \\
\cite{Dyhrberg.2016} & BTC, Federal funds rate, USD/EUR, USD/GBP, FTSE index, Gold futures, Gold cash. & 2010-2015, daily data. & BTC ranges in between a USD and a Gold. \\
\cite{AurelioF.Bariviera.2017} & BTC, USD/EUR, USD/GBP. & 2011-2017, daily data. 2013-2016, intraday data. & BTC presents large volatility and long-range memory (Hurst exponent higher than 0.5). Standard deviation of BTC is 10 times greater than other currencies. \\
\cite{Baur.2018} & BTC, Federal  funds rate, USD/EUR, USD/GBP, FTSE index, Gold futures, Gold cash. & 2010-2015, daily data. & BTC returns are not a hybrid of Gold and USD  returns. \\
\cite{GuglielmoMariaCaporale.2018} & BTC, LTC, Ripple, Dash & 2013-2017, daily data. & The four cryptocurrencies exhibit persistence (Hurst exponent higher than 0.5), yet the degree of persistence changes over time. \\
\cite{Hardle.2018} & BTC, XRP, LTC, ETH, Gold and S\&P500 & 2016-2018, daily data. & BTC, XRP, LTC, ETH exhibit higher volatility, skewness and kurtosis compared to Gold and S\&P500 daily returns. \\
\cite{Henriques.2018} & BTC and five exchange traded funds (ETFs): US equities (SPY), US bonds (TLT), US real estate (VNQ), Europe and Far East equities (EFA), and Gold (GLD). & 2011-2017, daily data. & BTC can be a substitute for Gold in an investment portfolio, achieving a higher risk adjusted return. \\
\cite{Jiang.2018} & BTC & 2010-2017, daily data. & Long-term memory and high degree of inefficiency ratio  exists in the BTC market.\\
\cite{Klein.2018} & BTC, CRIX index, Gold, Silver, crude oil prices for West Texas Intermediate (WTI), S\&P500 index, MSCI World and MSCI Emerging Markets 50 index. & 2011-2017, daily data. & BTC returns have the highest mean and standard deviation among all assets. \\
\cite{Selmi.2018} & BTC, Gold, Brent crude oil & 2011-2017, daily data. & Both BTC and Gold would serve the roles of a hedge, a safe haven and a diversifier for oil price movements. \\
\cite{Stosic.2018} & Top 119 cryptocurrencies. & 2016-2017, daily data. & Collective behaviour of the cryptocurrency market. \\
\cite{Takaishi.2018} & BTC, GBP/USD & 2014-2016, intraday data & The 1-min return distribution of BTC is fat-tailed, with high kurtosis; BTC time series exhibits multifractality. \\
\cite{Urquhart.2016} & BTC & 2010-2016, daily data. & Hurst statistic indicates strong anti-persistence (values lower than 0.5). \\
\cite{Wei.2018} & 456 different cryptocurrencies. & 2017, daily data. & Lower volatility for liquid cryptocurrencies. Illiquid cryptocurrencies exhibit strong return anti-persistence in the form of a low Hurst exponent. \\
\cite{Zhang.2018} & 70 \% of cryptocurrencies market. & 2013-2018, daily data. & Cryptocurrencies exhibit heavy tails, quickly decaying returns autocorrelations, slowly decaying autocorrelations for absolute returns, strong volatility clustering, leverage effects, long-range dependence, power-law correlation between price and volume. \\
\cite{Borri.2019} & BTC, ETH, LTC, XRP, Gold Bullion, the CBOE volatility index (VIX), the S\&P400 commodity chemicals index, and the S\&P500 index. & 2017-2018, daily data. & Cryptocurrencies exhibit large and volatile return swings, and are riskier than most of the other assets.\\
\hline \hline
\end{tabularx}}
\end{table}

Trough the means of dimensionality reduction techniques (like Factor Analysis) and classification techniques (like Binary Logistic Regression and Support Vector Machines), we prove that most of the variation among cryptocurrencies, stocks, exchanges rates and commodities can be explained by three factors: the tail factor, the moment factor and the memory factor.

Our results add to the findings from literature by showing that the most important factor which differentiates the cryptocurrencies from classical assets is the tail behaviour of the log-returns distribution. This finding is confirmed by the classical factor analysis, performed on a static basis and also by using the expanding window approach, where the assets universe is seen in an evolutionary dynamic. The most important result of our paper is the discovery of a phenotypic convergence of cryptocurrencies, compared to the classical assets. By using an expanding window approach, we are able to show that the cryptocurrencies have a convergent dynamic in relationship to the classical assets and this convergence is driven mainly by the tail behaviour of the log-returns distribution. More, the cryptocurrencies as a species exhibit a divergent evolution in relation to classical assets.  Originated from biology, the concept of divergent evolution refers to the accumulation of differences between related populations, leading to speciation \citep{Rieseberg.2004}. Divergent evolution is typically exhibited when two populations are exposed to different selective factors, driving their adaptation to the environment \citep{Bergstrom.2016}. A related analysis can be found in \cite{ElBahrawy.2017}, where the cryptocurrencies market is seen as an evolutive system, with several characteristics which are preserved over time. According to \cite{ElBahrawy.2017}, the evolution of the cryptocurrencies market has been ruled by "neutral" forces, i.e. no cryptocurrency has shown any strong selective advantage over the other.

The paper is subsequently organized as follows: the first section describes the statistical methodology used, including Factor analysis, Logistic regression, Support vector machines (SVM) and the evolutive divergence. The second section describes the data-set and interprets the results of the classification; the third section describes the phenotypic convergence, while the last section concludes. The codes used to obtain the results in this paper are available via \url{www.quantlet.de}. 